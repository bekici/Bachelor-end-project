\section{Stochastic optimization}
\subsection{Machine learning}
\subsubsection{Motivation}
Stochastic optimization is routinely used in machine learning. Machine learning is a field that enables computers to learn from data and improve their performance. Learning methodologies in machine learning generally fall under one of three primary paradigms: supervised learning, unsupervised learning, and reinforcement learning \cite[101-105]{Goodfellow-et-al-2016}. This thesis will focus on the approach of supervised learning. As suggested by the name, this method involves training a machine learning model using labeled data. The model is exposed to training examples composed of inputs and their corresponding desired outputs, which are provided by a human supervisor. The objective is to formulate a general rule that can effectively map these inputs to the correct outputs. In this context, the model is equipped with a set of parameters, commonly referred to as weights and biases, and denoted as $\theta \in \mathbb{R}^{p}$, which are randomly initialized before the training process begins. The key to achieving the goal of accurately mapping inputs to their intended outputs lies in identifying an optimal set of these parameters. Once the parameters are initialized, inputs are fed into the model, which in turn generates corresponding outputs. These outputs represent the model's predictions of the true values. Performance is measured using a loss function, which quantifies the difference between the predicted outputs and the true values. When dealing with an entire dataset, we calculate a cost or objective function, which is typically the average of all individual losses across the dataset. The aim of supervised learning is to adjust the parameters $\theta$ to minimize this cost function, thereby improving overall model performance. To achieve this, Stochastic Gradient Descent (SGD) methods are commonly employed, especially when dealing with large datasets. The details of these concepts and their implementation in the optimization of neural networks will be discussed in the subsequent sections.
\subsubsection{Basics of Supervised Learning and Neural Networks}\label{Basic_of_SL_FFNN}
\paragraph{Supervised Learning}
Supervised learning uses labeled data, consisting of input-output pairs $(x, y)$. The underlying relationship between the inputs and outputs is often represented by a 'true function' $f: \mathbb{R}^{d_{1}}\rightarrow \mathbb{R}^{d_{2}}$, where $d_{1}, d_{2} \in \mathbb{Z}_{+},$ such that $f(x)=y$ \cite{bishop2006}. However, in practice, this function is unknown, and our goal is to approximate it as closely as possible using a model. In the context of supervised learning, a model is represented by a function $f_{m}$ which takes an input $x$ and parameters $\theta$ to produce a predicted output $\hat{y}=f_{m}(x, \theta)$ \cite{Goodfellow-et-al-2016}. To approximate $f,$ the model $f_{m},$ is trained on a subset of the data that is available, $\{(x^{(1)},y^{(1)}),\ldots,(x^{(n)},y^{(n)})\}$, where $x^{(i)}$ is an input and $y^{(i)}$ is the corresponding label we wish to get from $f(x^{(i)}) \text{ } \text{ }\forall i\in\{1,\ldots n\}$. The parameters $\theta$ are adjusted such that the model's outputs $\hat{y}$ closely align with the true outputs $y$ from the training data. This alignment is quantified by a cost function, which measures the average discrepancy between the model's predictions and the actual outputs over the entire dataset. The goal of the training process is not only to minimize this cost on the training data, but, more importantly, to modify the model such that it generalizes well to unseen data. This is known as the model's capacity for generalization. The better a model can generalize, the more accurate its predictions will be when presented with new data outside of the training set \cite{james2013}.

\paragraph{Feed-forward Neural Networks}\label{Basic_of_FFNN}
Feed-forward neural networks are a common type of model used in supervised learning. They attempt to approximate the unknown function $f: x \mapsto y$ by defining a function $f_{m}: (x,\theta) \mapsto \hat{y}$, where $\hat{y}$ is the predicted output, $x$ is the input, and $\theta$ represents the model's parameters (the weights and biases of the neurons in the network). The network consists of multiple layers, including input, hidden, and output layers. Each layer contains a number of neurons that receive inputs from the previous layer, apply a weighted sum with a bias term, and then apply an activation function. This activation function introduces non-linearity into the model, enabling it to learn and represent complex patterns in the data. The training process involves adjusting the parameters $\theta$ to minimize the cost function. This is done using an optimization algorithm, such as stochastic gradient descent, which iteratively updates the weights and biases based on the gradients of the loss function with respect to these parameters. A detailed mathematical exploration of stochastic gradient descent will be provided later in the document. Afterwards, the function of each layer in a feed-forward neural network will be discussed, particularly within the context of a specific application of these networks.
\newpage
\subsection{Minimizing Sums and Expectation}
In the context of supervised learning, we aim to approximate the unknown function $f: \mathbb{R}^{d_{1}}\rightarrow \mathbb{R}^{d_{2}}$. This function maps each input $x$ to an output $y$. We approximate $f$ using a model function, denoted $f_{m}: \mathbb{R}^{d_{1}}\times \mathbb{R}^{p}\rightarrow \mathbb{R}^{d_{2}}$. The model function $f_{m}$ takes an input $x$ and a set of parameters $\theta$ and produces an output $\hat{y}$, which is our prediction of the true output $y$. We have at our disposal an initialized set of parameters $\theta$ and a training set of $n$ examples $T=\{(x^{(1)},y^{(1)}),\ldots,(x^{(n)},y^{(n)})\},$ where $x^{(i)} \in \mathbb{R}^{d_{1}}$ and $y^{(i)} \in \mathbb{R}^{d_{2}} \text{ } \text{ } \forall i\in\{1,\ldots,n\}.$ We assume that there is an unknown joint probability distribution $P_{XY}:\Omega\rightarrow [0,1],$ where $\Omega = \mathbb{R}^{d_{1}} \times \mathbb{R}^{d_{2}}$ is the sample space. Furthermore, we assume that the training examples $(x^{(1)},y^{(1)}),\ldots,(x^{(n)},y^{(n)})$ are drawn i.i.d. from $P_{XY}.$ We also assume that we are given a non-negative real-valued loss function $L(\hat{y},y),$ which measures the difference between the prediction $\hat{y}$ and the true outcome $y.$ As discussed in \cite[20]{coursenotesML}, the cost $F$ associated with the function $f_{m}$ is then defined as an expectation of the loss function $L$. To approximate $f$ by $f_{m},$ we need to minimize the cost function. The optimization problem can then be expressed as:  
\begin{equation*}\label{min_expectations}\tag{5.2.1}
\underset{\theta \in \mathbb{R}^{p}}{\text{min}} F(\theta) \overset{\text{def.}}{=} \mathbb{E}_{(x,y)\sim P_{XY}}[L(f_{m}(x, \theta), y)] = \int_{\Omega} L(f_{m}(x, \theta), y) dP_{XY}(x,y).
\end{equation*}
While minimizing expectations is a general framework that allows us to account for uncertainty in the data, the cost function $F$ cannot be computed, because the joint distribution $P_{XY}$ is unknown to the learning algorithm. However, we can calculate an approximation to the cost function, known as the empirical cost and denote it by $F_{m}$. This is achieved by averaging the values of the loss function across the entire training set. The optimization problem can then be expressed as:
\begin{equation*}\label{largesums}\tag{5.2.2}
\underset{\theta \in \mathbb{R}^{p}}{\text{min}} F_{m}(\theta) \overset{\text{def.}}{=} \frac{1}{n} \sum_{i=1}^{n}L(f_{m}(x^{(i)}, \theta), y^{(i)}).
\end{equation*}
Problem \eqref{largesums} is a special case of \eqref{min_expectations} when replacing the true distribution $P_{XY}:\Omega\rightarrow[0,1]$ with the joint Empirical Cumulative Distribution (ECDF) $\hat{P}_{XY_{n}}:T\rightarrow[0,1],$ defined by the training set $T.$
\begin{definition}(\cite[1]{CastroNotes})\label{ECDF}
Let $(X^{(1)},Y^{(1)}),\ldots,(X^{(n)},Y^{(n)})$ be a sample of $n$ data points, where each $X^{(i)} = (X^{(i)}_{1},\ldots,X^{(i)}_{d_{1}})$ and $Y^{(i)} = (Y^{(i)}_{1},\ldots,Y^{(i)}_{d_{2}})$ are vector-valued random variables. The sequence $\left\{(X^{(i)},Y^{(i)})\right\}_{i=1}^{n}$ is assumed to be independent and identically distributed (i.i.d.) according to the joint distribution $P_{XY}.$ We denote, $P_{XY}(x,y)=\mathbb{P}\left(X_{1}\leq x_{1},\ldots,X_{d_{1}}\leq x_{d_{1}},Y_{1}\leq y_{1},\ldots,Y_{d_{2}}\leq y_{d_{2}}\right).$ 
The joint ECDF for $P_{XY}$ is defined as
\begin{equation*}\tag{5.2.3}
\hat{P}_{XY_{n}}(x,y) = \frac{1}{n} \sum_{i=1}^{n} \mathbbm{1}\left\{X_{1}^{(i)}\leq x_{1},\ldots,X_{d_{1}}^{(i)}\leq x_{d_{1}},Y_{1}^{(i)}\leq y_{1},\ldots,Y_{d_{2}}^{(i)}\leq y_{d_{2}}\right\}.\footnote{Here, $\mathbbm{1}$ is the indicator function, which is one if the condition inside the brackets is true and zero otherwise.}
\end{equation*}
\end{definition}
The joint ECDF takes the form of a step function, which by jumps by $1/n$ at each of the observed data points $(X^{(i)}, Y^{(i)})$ Rui Castro states in his notes about the univariate ECDF, "Clearly the ECDF seems to be a sensible estimator for the underlying distribution."\cite{CastroNotes} The following theorem provides a justification to use $\hat{P}_{n}$ as an estimator for $P$ in the univariate case:
\begin{theorem}[Glivenko-Cantelli Lemma]\textnormal{(\cite[2]{CastroNotes})}\label{Glivenko}
For a sequence of univariate random variables, the empirical distribution converges uniformly to $P(x),$ namely
\begin{equation*}\tag{5.2.4}
\left\Vert \hat{P}_{n} - P\right\Vert_{\infty} = \underset{x\in \mathbb{R}}{\sup}\left|\hat{P}_{n}(x) - P(x)\right|\xrightarrow{a.s.} 0,
\end{equation*}
as $n\rightarrow\infty,$ where the superscript a.s. denotes convergence almost surely.
\end{theorem}
For a comprehensive proof of this theorem, please refer to Castro's work \cite[8]{CastroNotes}. Our data involves multivariate ECDFs, therefore we have to seek for a multidimensional extension of the Glivenko-Cantelli theorem. Specifically, the Vapnik-Chervonenkis inequality serves as a suitable generalization. The theorem and its proof is presented in an abstract form in \cite[4-5]{kahle2006}. We will reformulate this theorem into a corollary that aligns more closely with our particular case. 
\begin{corollary}[Vapnik-Chervonenkis]\label{VC}
Let $P_{XY}$ be an unknown distribution, $\hat{P}_{XY_{n}}$ the joint ECDF and $\Omega=\mathbb{R}^{d_{1}}\times \mathbb{R}^{d_{2}}$ the sample space. Then for any $\epsilon>0,$
\begin{equation*}\tag{5.2.5}
\mathbb{P}\left\{\underset{(x,y)\in\Omega}{\sup}\left|P_{XY}(x,y)-\hat{P}_{XY_{n}}(x,y)\right| > \epsilon\right\} \xrightarrow{n\rightarrow\infty} 0
\end{equation*}
\end{corollary}
Corollary \ref{VC} suggests that as long as the size of the training dataset $n$ is sufficiently large, the joint ECDF should be a good enough estimator for the underlying distribution $P_{XY}.$ In this setting, the expectation under $\hat{P}_{XY_{n}}$ of the observations $(x,y)$ should be a good estimate of the true expectation under the underlying distribution $P_{XY}$ of the random vectors $(x,y).$ The empirical distribution, considers a finite set of observed data points $(x,y)$, which are treated as if they were discrete. Each observed value is assigned a probability of $1/n.$ With this knowledge, the equation given in \eqref{largesums} can be derived from the equation given in \eqref{min_expectations} as follows:
\begin{equation*}\tag{5.2.6}
\begin{aligned}
F(\theta)
    &= \mathbb{E}_{(x,y)\sim P_{XY}}[L(f_{m}(x, \theta), y)] \\
    &\approx \mathbb{E}_{(x,y)\sim \hat{P}_{XY_{n}}}[L(f_{m}(x, \theta), y)]\\ 
    &= \sum_{i=1}^{n}\mathbb{P}((x, y) = (x^{(i)}, y^{(i)}))\cdot L(f_{m}(x^{(i)}, \theta), y^{(i)})\\ 
    &= \frac{1}{n}\sum_{i=1}^{n}L(f_{m}(x^{(i)}, \theta), y^{(i)})\\
    &= \frac{1}{n}\sum_{i=1}^{n}L_{i}(f_{m}(x, \theta), y),
\end{aligned}
\end{equation*}
where we've introduced the last equality as a shorthand notation for convenience. 
Minimizing sums with a discrete empirical distribution is a widely used technique in machine learning. In this setting, each data point is treated as equally likely. This formulation conveniently transforms the cost function, initially defined as an expectation, into a sum, which simplifies the problem and makes it easier to work with. In the subsequent sections, we will discuss Batch Gradient Descent (BGD) and Stochastic Gradient Descent (SGD). These methods are designed to solve the minimization problems presented in Equations \eqref{min_expectations} and \eqref{largesums}. For ease of notation, we will use $z$ as a substitute for $(x,y)$ with $z\sim P_z$, and simplify the loss function $L(f_{m}(x, \theta), y)$ to $L(\theta, z)$.
\subsection{Batch Gradient Descent (BGD)}
We have studied the usual deterministic (batch) gradient descent in extensive detail in Section \ref{sec: GD}. In machine learning, we update the parameters, $\theta\in\mathbb{R}^{p},$ using an iterative process. Each iteration can be represented by the following equation:
\begin{equation*}\tag{5.3.1}
\theta_{k+1} = \theta_{k} - \tau_{k}\nabla F(\theta_{k})
\end{equation*}
In this equation, $\theta_{k+1}$ represents the updated parameters, while $\theta_{k}$ represents the current parameters, and $\nabla F(\theta_{k})$ is the gradient of the function $F$ at $\theta_{k}$. In the absence of a strong convexity assumption and using the fixed step size method, the step size, $\tau_{k}$, should be chosen such that $0 < \tau_{k} < \tau_{\text{max}}$, where $\tau_{\text{max}} = 1/L$. Here, $L$ represents the Lipschitz constant of the gradient $\nabla F$. Choosing $\tau \approx \tau_{max}$ ensures quite fast convergence. (even linear rates if $F$ is strongly convex.) Most minimization problems in machine learning, are problems in which the cost function is non-convex. In this context, the step size $\tau$ requires careful tuning through experimentation. The goal is to locate a local minimum that is sufficiently close to a global minimum. Suppose we have a training set of data, $\{z^{(1)}\ldots,z^{(n)}\},$ available. Then the cost function can be expressed as a large sum with gradient illustrated as: 
\begin{equation*}\label{full_grad_of_sum}\tag{5.3.2}
\nabla F(\theta) = \frac{1}{n} \sum_{i=1}^{n}\nabla L_{i}(\theta, z)
\end{equation*}
Computing this gradient typically has complexity $O(np)$ if computing $\nabla L_{i}(\theta, z)$ has linear complexity $p.$ Here $i \in \{1,\ldots,n\}$ represent indices and not random variables. If the size of the dataset $n$ is not too large, then one can afford to perform a few deterministic iterations. Then deterministic methods can be faster. But if $n$ is too big for even one deterministic iteration, stochastic methods can help because they can break down the cost of deterministic iterations into smaller pieces, which allows for a more precise approach.
\newpage
\subsection{Stochastic Gradient Descent (SGD)}\label{SGD section}
In contrast to batch gradient descent, which requires the computation of the full gradient $\nabla F$, the SGD algorithm approximates the gradient using a single stochastic estimate, $\nabla L(\theta, z)$. Consider a random variable $z\sim P_z$ and sample space $\Omega.$ The objective is to minimize the expected value of $L(\theta, z),$ as given in Equation \eqref{min_expectations}. SGD starts from an initial point $\theta_{0}.$ Then, for each iteration index $k,$ SGD updates are performed using stochastic gradient estimates, computed based on specific draws $z_{k}\sim P_{z}.$ The update equation is as follows:
\begin{equation*}\tag{5.4.1}
\theta_{k+1} = \theta_{k} - \tau_{k} \nabla L(\theta_{k}, z_{k})
\end{equation*}
In this equation, $\theta_{k}$ is the current parameter estimate, $\tau_k$ is the step size at iteration $k$, and $\nabla L(\theta_k, z_k)$ is the stochastic gradient estimate based on the draw $z_k$ from the distribution $P_z$.
Under certain regularity conditions\footnote{These conditions, which stem from Lebesgue's Dominated Convergence Theorem \cite[67]{stein2005real}, include: $\textnormal{\RN{1}}$: For almost all $z \in \Omega$, $\nabla L(\theta, z)$ exists for all $\theta\in\mathbb{R}^{p}$. $\textnormal{\RN{2}}$: $L(\theta, z)$ is a Lebesgue-integrable function of $z$ for all $\theta\in\mathbb{R}^{p}$. $\textnormal{\RN{3}}$: There exists a function $\phi : \Omega \rightarrow \mathbb{R}$ such that $|\nabla L(\theta, z)| \leq \phi(z)$ for all $\theta\in\mathbb{R}^{p}$ and almost all $z \in \Omega$.} that allow for interchanging limit operations such as differentiation, the stochastic gradient $\nabla L(\theta, z)$ is an unbiased estimator of the full gradient. The derivation of this proceeds as follows:
\begin{equation*}\label{gd_unbiased}\tag{5.4.2}
\mathbb{E}_{z \sim P_z}[\nabla L(\theta, z)] = \int_{\Omega}\nabla L(\theta,z)dP_{z}(z) = \nabla\int_{\Omega}L(\theta,z)dP_{z}(z) = \nabla\mathbb{E}_{z \sim P_z}[L(\theta, z)] = \nabla F(\theta)
\end{equation*}
The expectation is taken over all possible draws of the random variable $z.$  It demonstrates that the expected value of the stochastic gradient computed for any draw $z_k$ from the distribution $P_z$ is equal to the true gradient $\nabla F(x)$.
\begin{comment}
Starting from an initial point $\theta_0$, the SGD updates are performed using the stochastic gradient estimate, computed based on the specific draw $z_k$ for each iteration index $k$. The update equation is as follows:
\begin{equation*}\tag{4.4.2}
\theta_{k+1} = \theta_k - \tau_k \nabla L(\theta_k, z_k)
\end{equation*}
In this equation, $\theta_k$ is the current parameter estimate, $\tau_k$ is the step size at iteration $k$, and $\nabla L(\theta_k, z_k)$ is the stochastic gradient estimate based on the draw $z_k$ from the distribution $P_z$.  
\end{comment}
In a machine learning setting, where $P_z$ is unknown and we have access to a set of training data, $\{z^{(1)},\ldots,z^{(n)}\}$, we generally approximate $P_z$ by the empirical cumulative distribution function $\hat{P}_{n}$ (Definition \ref{ECDF}). The minimization problem is given in Equation \eqref{largesums}.
\begin{comment}
\begin{equation*}\tag{4.4.4}
\underset{\theta\in\mathbb{R}^{p}}{\min}F(\theta) = \frac{1}{n}\sum_{i=1}^{n}L(\theta,z^{(i)}) = \frac{1}{n}\sum_{i=1}^{n}L_{i}(\theta,z)    
\end{equation*}
\end{comment}
Starting from an initial point $\theta_{0},$ the SGD updates are performed using stochastic gradient estimates, $\nabla L(\theta,z^{(i)})=L_{i}(\theta,z)$. The update equation is as follows: 
\begin{equation*}\label{SGD_update}\tag{5.4.3}
\theta_{k+1} = \theta_k - \tau_k \nabla L_{i(k)}(\theta_k,z_{k})
\end{equation*}
In this case, $i(k)$ is a random variable drawn from the discrete uniform distribution on the index set $\{1, \ldots, n\}$. For each iteration index $k$, $i(k)$ corresponds to the observation $z_{k}^{(i)}$, and the gradient computation is based on this specific observation. Here, the stochastic gradient $\nabla L(\theta, z)$ is again an unbiased estimator of the full gradient, as demonstrated via the following computations: 
\begin{equation*}\tag{5.4.4}
\mathbb{E}_{z \sim \hat{P}_{n}}[\nabla L(\theta, z)] = \sum_{i=1}^{n}\mathbb{P}(z = z^{(i)})\cdot \nabla L(\theta, z^{(i)}) =  \nabla\frac{1}{n}\sum_{i=1}^{n}L(\theta, z^{(i)}) = \nabla\mathbb{E}_{z \sim \hat{P}_{n}}[L(\theta, z)] = \nabla F(\theta)
\end{equation*}
 The sequence of vectors $\theta_{k+1}$ forms a random process, and the convergence analysis of SGD aims to determine whether this process converges in quadratic mean to a deterministic vector $\theta^*$ that minimizes the objective function $F$.
\begin{comment}
Note that in the case of minimizing large sums, batch gradient descent has complexity $O(np),$ while a step of SGD only has complexity $O(p).$ SGD is thus advantageous when $n$ is very large, and one cannot afford to compute the full gradient.
\end{comment}
An essential aspect of SGD is determining an appropriate step size. One option is to choose a fixed step size. A commonly used step size is $\tau=0.01.$ This value, is low enough to prevent overshooting the minimum, but high enough to ensure a reasonable convergence time. Bengio (2012) suggests, "A default value of 0.01 typically works for standard multi-layer neural networks but relying exclusively on this default value would be unwise."\cite{bengio2012practical} A question that naturally arises is whether the optimal step size can be calculated a priori. The authors, Reed and Marks note, "... in general, it is not possible to calculate the best learning rate a priori."\cite{reed1999neural} Hence, finding a good step size typically requires trial and error. One can also opt to use adaptive step size methods. These methods can adjust to the characteristics of the dataset based on the gradient behavior. The step size $\tau_k$ should go to zero as $k \rightarrow \infty$, but not too quickly, as it can impact the convergence of the algorithm. As extensively covered in \cite[9-11]{murata}, a typical step size schedule that ensures both properties is to have $\tau_k \sim k^{-1}$ for $k \rightarrow \infty$. 

\newpage
\subsection{Convergence Analysis of Stochastic Gradient Descent}
\subsubsection{Convergence Analysis of SGD for a cost formulated as a finite sum} 
To solve the minimization problem described in \eqref{largesums} by means of SGD, it is required that $\left\Vert \theta_{k} - \theta^{*}\right\Vert$ tends to zero in quadratic mean, or equivalently, $\mathbb{E}[\left\Vert \theta_{k} - \theta^{*}\right\Vert^{2}] \rightarrow 0$ as $k \rightarrow \infty,$ where $\left\Vert\cdot\right\Vert$ is the Euclidean norm. Proposition \ref{conv_prop_sgd} provides a theoretical foundation in which SGD is ensured to converge. 
\begin{proposition}\label{conv_prop_sgd}
Let $f: \mathbb{R}^{p} \longrightarrow \mathbb{R}$ be $\mu$-strongly convex and differentiable on $\textbf{dom}(f) = \mathbb{R}^{p}$, and is such that $\left\Vert\nabla f_{i}(\theta)\right\Vert^{2} \leq C^{2}.$ For the step size choice $\tau_{k} = \frac{1}{\mu (k+1)}$, one has
\begin{equation*}\label{bound_conv_SGD}\tag{5.5.1.1}
\mathbb{E}[\left\Vert \theta_{k} - \theta^{*}\right\Vert^{2}] \leq \frac{R}{k+1}\text{ }\text{ } where \text{ }\text{ } R = \textnormal{max}\left\{\left\Vert \theta_{0} - \theta^{*}\right\Vert^{2},\text{ } C^{2}/\mu^{2}\right\},
\end{equation*}
where $\mathbb{E}$ indicates an expectation with respect to the i.i.d. sampling performed at each iteration.
\end{proposition}
\begin{proof} By strong convexity, one has
\begin{align}
    &f(y) - f(x) \geq \langle \nabla f(x),\text{ } y-x\rangle + \frac{\mu}{2}||y-x||^{2} \text{ }\text{ } \forall x,y \in \mathbb{R}^{p} \text{,} \text{ } \text{ } \text{hence } \nonumber\\
    &f(\theta_{k}) - f(\theta^{*}) \geq \langle \nabla f(\theta^{*}),\text{ } \theta_{k}-\theta^{*}\rangle + \frac{\mu}{2}||\theta_{k}-\theta^{*}||^{2}\nonumber\\
    &f(\theta^{*}) - f(\theta_{k}) \geq \langle \nabla f(\theta_{k}),\text{ } \theta^{*}-\theta_{k}\rangle + \frac{\mu}{2}||\theta^{*}-\theta_{k}||^{2}.\nonumber
\end{align}
Summing these two inequalities and using $\nabla f(\theta^{*}) = 0$ leads to
\setcounter{equation}{0}
\begin{align}
    &\langle \nabla f(\theta_{k}) - \nabla f(\theta^{*}),\text{ } \theta_{k} - \theta^{*}\rangle = \langle \nabla f(\theta_{k}),\text{ } \theta_{k} - \theta^{*}\rangle \geq \mu ||\theta_{k} - \theta^{*}||^{2}.
\end{align}
Considering only the expectation with respect to the random sample of $i(k) \sim \textbf{i}_{k}$, one has
\begin{align}
\mathbb{E}_{\textbf{i}_{k}}[||\theta_{k+1}-\theta^{*}||^{2}]
    &= \mathbb{E}_{\textbf{i}_{k}}[||\theta_{k} - \tau_{k}\nabla f_{\textbf{i}_{k}}(\theta_{k}) - \theta^{*}||^{2}]\nonumber\\
    &= \mathbb{E}_{\textbf{i}_{k}}[||\theta_{k}-\theta^{*}||^{2} + 2\tau_{k}\langle\nabla f_{\textbf{i}_{k}}(\theta_{k}),\text{ } \theta^{*} - \theta_{k}\rangle + \tau_{k}^{2}||\nabla f_{\textbf{i}_{k}}(\theta_{k})||^{2}]\nonumber\\
    &= ||\theta_{k}-\theta^{*}||^{2} + 2\tau_{k}\langle \mathbb{E}_{\textbf{i}_{k}}[\nabla f_{\textbf{i}_{k}}(\theta_{k})], \text{ } \theta^{*} - \theta_{k}\rangle + \tau_{k}^{2} \mathbb{E}_{\textbf{i}_{k}}[||f_{\textbf{i}_{k}}(\theta_{k})||^{2}]\\
    &\leq ||\theta_{k}-\theta^{*}||^{2} + 2\tau_{k}\langle \nabla f(\theta_{k}), \text{ } \theta^{*} - \theta_{k}\rangle + \tau_{k}^{2} C^{2}\\
    &\leq ||\theta_{k}-\theta^{*}||^{2} - 2\mu \tau_{k} ||\theta_{k}-\theta^{*}||^{2} + \tau_{k}^{2} C^{2}.
\end{align}
To go from step (2) to (3), the fact that the gradient is unbiased given in \eqref{gd_unbiased} and the assumption that $\left\Vert\nabla f_{i}(\theta)\right\Vert^{2} \leq C^{2}$ is used. To go from step (3) to (4), inequality (1) is used. Taking now the full expectation with respect to all the other previous iterates, one obtains
\begin{align} 
    &\mathbb{E}[||\theta_{k+1}-\theta^{*}||^{2}]\leq \mathbb{E}[||\theta_{k}-\theta^{*}||^{2}] - 2\mu \tau_{k} \mathbb{E}[||\theta_{k}-\theta^{*}||^{2}] + \tau_{k}^{2} C^{2} = (1 - 2\mu \tau_{k}) \mathbb{E}[||\theta_{k}-\theta^{*}||^{2}] + \tau_{k}^{2} C^{2}\nonumber
    \hspace{0.075cm} \text{\scalebox{0.925}{(5)}}
\end{align}
By induction, the bound in \eqref{bound_conv_SGD} can be shown. Denote $\varepsilon_{k} \overset{\scalebox{0.5}{def.}}{=} \mathbb{E}[||\theta_{k}-\theta^{*}||^{2}]$. Indeed, for $k = 0$, it holds $$\varepsilon_{0} = \mathbb{E}[||\theta_{0}-\theta^{*}||^{2}] \leq \frac{\textnormal{max}\left\{\left\Vert \theta_{0} - \theta^{*}\right\Vert^{2},\text{ } C^{2}/\mu^{2}\right\}}{1} = \frac{R}{1}$$
The induction hypothesis reads: $\varepsilon_{k} \leq \frac{R}{k+1}.$ Using inequality (5) with $\tau_{k} = \frac{1}{\mu(k+1)},$ and denoting $m = k+1$
\begin{align}
\varepsilon_{k+1}
    &\leq (1-2\mu \tau_{k}) \varepsilon_{k} + \tau_{k}^{2} C^{2} = \left(1-\frac{2}{m}\right)\varepsilon_{k} + \frac{C^{2}}{(\mu m)^{2}}\nonumber\\
    &\leq \left(1 - \frac{2}{m}\right)\frac{R}{m} + \frac{R}{m^{2}} = \left(\frac{1}{m} - \frac{1}{m^{2}}\right)R = \frac{m-1}{m^{2}}R = \frac{m^{2} - 1}{m^{2}}\frac{1}{m + 1}R \leq \frac{R}{m+1}\nonumber
\end{align}
Hence the bound holds for all $k \in \mathbb{N}.$ Now since, $0 \leq \mathbb{E}[\left\Vert \theta_{k} - \theta^{*}\right\Vert^{2}] \leq \frac{R}{k+1} \xrightarrow{k \rightarrow 0} \infty$. It follows from the squeeze theorem\footnote{See \ref{squeeze_theorem} for the squeeze theorem.} that $\mathbb{E}[\left\Vert \theta_{k} - \theta^{*}\right\Vert^{2}] \xrightarrow{k \rightarrow \infty} 0.$ In other words, $(\theta_{k})_{k\in\mathbb{N}}$ converges on the $L^{2}$ norm for random variables with $\theta_{k} \xrightarrow{L^{2}} \theta^{*}.$ 
\end{proof}
\subsubsection{Convergence Analysis of SGD for a cost formulated as an integral}
In the general setting with problem \eqref{min_expectations}, where $(\Omega,\mathcal{F}, \mathbb{P})$ is a probability space, $L:\mathbb{R}^{p}\times\Omega\rightarrow \mathbb{R}$ a loss function depending on a parameter $\theta$ and random argument $z$ with $z\sim P_{z},$ where $P_{z}$ the the underlying distribution, the goal of SGD is to find a minimum of $F$. It operates iteratively using the update formula:
\begin{align}
\theta_{k+1} = \theta_{k} - \tau_{k}\nabla_{\theta}L(\theta_{k}, z_{k}), \tag{1}
\end{align}
where at iteration $k$:
\begin{itemize}
\item $\tau_k$ is a deterministic "step size schedule,"
\item $z_k \in \Omega$ is a random variable independent of any previously drawn random variables.
\end{itemize}
To prove the convergence, the following hypotheses are needed:
\begin{enumerate}
\item The gradient of $L$ is bounded:
\begin{align}
\exists B > 0\colon \underset{z\in \Omega}{\sup}\left\Vert\nabla_{\theta} L(\theta, z)\right\Vert^{2} \leq B, \hspace{0.2cm} \forall \theta \in \mathbb{R}^{p}. \tag{2}
\end{align}
\item $F$ is strongly convex:
\begin{align}
\exists\mu > 0: F(y) \geq F(x) + \langle\nabla F(x), y-x\rangle + \frac{\mu}{2}\left\Vert x-y\right\Vert^{2}, \hspace{0.2cm} \forall x, y \in \mathbb{R}^{p}. \tag{3}
\end{align}
\end{enumerate}
Note that for $\mu = 0$, this is just the usual convexity, i.e., the function is above its tangent. For $\mu > 0$, the function is above a parabola. Proving the convergence of $(\theta_{k})_{k\in \mathbb{N}}$ involves verifying each item in the corresponding proposition.
\begin{proposition}[\cite{turinici2021convergence}]\label{pf_sgd_conv_general}
Suppose that each $L(\cdot, z)$ is differentiable and that $F$ satisfies hypotheses (2) and (3). Then:\

the function $F$ has a unique minimum $\theta^{*}$;\
For any $k \geq 0$ denote
\begin{align}
d_{k} = \mathbb{E}[\left\Vert \theta_{k}-\theta^{*}\right\Vert^{2}]. \tag{4}
\end{align}
Then
\begin{align}
d_{k+1} \leq (1 - \tau_{k}\mu)d_{k} + \tau_{k}^{2}B. \tag{5}
\end{align}
For any $\epsilon > 0$ there exists a $\tau > 0$ such that if $\tau_{k} = \tau$ then
\begin{align}
\limsup_{k\rightarrow\infty}{\mathbb{E}[\left\Vert \theta_{k+1}-\theta^{*}\right\Vert^{2}]} \leq \epsilon. \tag{6}
\end{align}
Take sequence $(\tau_{k})_{k\in \mathbb{N}}$ such that:\
\begin{align}
\tau_{k} \rightarrow 0 \text{ } and \text{ } \sum_{k\geq 1} \tau_{k} = \infty. \tag{7}
\end{align}
Then $d_{k} = \mathbb{E}[\left\Vert \theta_{k}-\theta^{*}\right\Vert^{2}] \xrightarrow{k\rightarrow\infty} 0,$ that is $\lim_{k\rightarrow\infty}\theta_{k} = \theta^{*},$ where the convergence is on the $L^{2}$ norm for random variables (convergence in quadratic mean), In other words, $\theta_{k} \xrightarrow{L^{2}} \theta^{*}$.
\end{proposition}
For a comprehensive proof of this proposition, please refer to Turinici's work \cite{turinici2021convergence}.
\newpage
\subsection{Overview of Stochastic Global Optimization Algorithms}
\subsubsection{Introduction}
In the context of global optimization methods, the goal of solving the unconstrained minimization problem described in \eqref{eq:8} remains the same. While local optimization methods like gradient descent are effective for convex functions, they can struggle with non-convex functions due to their reliance on gradient information, which may lead them to become trapped in local minima. Stochastic Gradient Descent (SGD) is often sufficient in machine learning settings where a local minimum can provide a satisfactory solution and balance between fitting the data and generalizing to new examples, as discussed in \cite[282-290]{Goodfellow-et-al-2016}.

Global optimization methods serve as an alternative to gradient-based optimization techniques, aiming to find a global minimum. These methods are especially important for solving non-convex functions where local optimization methods may not perform well. In contrast to local optimization methods, global optimization methods employ various search strategies to explore the solution space more thoroughly, thereby increasing the likelihood of identifying the global optimum, as discussed in \cite[1-5]{horst1995handbook}. In this section, two popular global optimization methods: Simulated Annealing and Particle Swarm Optimization (PSO) will be discussed.

\subsubsection{Simulated Annealing}
\paragraph{History and Motivation}
The Simulated Annealing algorithm is a stochastic global optimization method that searches for the global minimum of a non-convex function. These functions typically arise from problems in physics, chemistry and biology and are called energy functions \cite[1]{XIANG1997216}. The key feature of simulated annealing is that it provides a means to escape local minima by allowing hill-climbing moves (i.e., moves which worsen the objective function value) in hopes of finding a global minimum \cite{Henderson2003}. The analogy of simulated annealing is that it is similar to the process of blacksmiths making a sword. In order to make a strong and sharp sword, the molecules of the metal must be aligned in the same way. Blacksmiths achieve this by repeatedly heating and cooling the metal to give the molecules a chance to realign themselves into the lowest energy state. This is called annealing. Similarly, the simulated annealing optimization algorithm finds the optimal solution by randomly exploring the solution space and gradually reducing the "temperature" or the randomness in the search process to converge to the optimal solution. So, just as blacksmiths use annealing to create a strong sword, simulated annealing uses a similar approach to find the optimal solution in a search space. 
\paragraph{General Setting}
To describe the features of the simulated annealing algorithm, several definitions are needed. These definition are for the most part identical to the definitions used in gradient-based optimization methods. Let $\left(\Omega,\left\Vert\cdot\right\Vert_{\Omega}\right)$ be the solution space (i.e., the set of all possible solutions). Let $f: \Omega \rightarrow \mathbb{R}$ be an objective function (possibly non-convex) defined on the solution space. The goal is to find an optimal solution, $x^{*}$ such that $f(x^{*}) \leq f(x) \text{ }\forall x \in \Omega.$ The objective function must be bounded to ensure that $x^{*}$ exists. Furthermore, define $S = \{x + \varepsilon\cdot\xi \text{ }|\text{ } \xi_{i} \in \left(-1, 1\right) \text{ } \forall i\}$ to be the neighborhood function for $x \in \Omega$ where $\varepsilon > 0$ is arbitrary. So, associated with every solution, $x \in \Omega$, are neighboring solutions in $S$ that can be reached in a single iteration of the algorithm \cite[2]{metropolis1953equation}. The set $S$ represents a region in $\Omega$ obtained by adding small displacements to $x$ along each coordinate axis. In case $\Omega = \mathbb{R}^n$, the set represents an n-dimensional hypercube with $x$ as its origin and side lengths $2\varepsilon.$ From now on we assume that $\Omega = \mathbb{R}^n.$ Simulated annealing starts with a starting vector $x \in \Omega.$ A neighboring solution $x' \in S$ is then generated using the rule described in \cite[2]{metropolis1953equation}. 
Namely according to the following pre-specified rule:
\begin{equation}\tag{5.6.2.1}
x' = x + \varepsilon\cdot\xi  
\end{equation}
where $\varepsilon$ is the maximum allowed displacement, which for the sake of this argument is arbitrary, and $\xi$ is a random vector with $\xi_{i} \in \left(-1,1\right)\text{ } \forall i \in \{1,\ldots,n\}.$ Simulated annealing is based on the Metropolis acceptance criterion \parencite[2]{metropolis1953equation}, which models how a thermodynamic system moves from the current state $x$ to the next state $x'$, intending to minimize the energy content. The candidate $x'$ is accepted as the new solution based on the acceptance probability provided in \cite[289]{Henderson2003}:
\begin{equation}\tag{5.6.2.2}
\mathbb{P}\{\text{Accept $x'$ as next solution}\} = 
    \begin{cases}
        \exp\left[-\left(f(x')-f(x)\right)/t_{k}\right] \hspace{0.5cm}\text{if $f(x')-f(x) > 0$}\\
        1 \text{\hspace{4.1cm} if $f(x')-f(x) \leq 0$}
    \end{cases}
\end{equation}
Define $t_{k}$ as the temperature parameter at iteration $k$, such that
\begin{equation}\tag{5.6.2.3}
t_k > 0 \hspace{0.33em}\text{for all}\hspace{0.33em} k \hspace{0.33em}\text{and}\hspace{0.23em}\lim_{k\rightarrow\infty}t_{k} = 0
\end{equation}
Now as $t_{k}$ decreases $\exp\left[-\left(\Delta f\right)/t_{k}\right]$ decreases as well, lowering the probability that new solutions ($x_{new}$) are accepted in case  $\Delta f = f(x_{new})-f(x_{old}) > 0.$ In other words, when the temperature is lower, the algorithm is less likely to accept a worse solution. If the temperature schedule $t_{k}$ is chosen such that temperature reduction is sufficiently slow, a steady progression towards an optimal solution can be achieved \cite[289]{Henderson2003}. A typical temperature schedule is $t_{k+1} = \alpha t_{k}$ $\implies$ $t_{k} = \alpha^{k} t_{0},$ where $t_{0} > 0$ is the chosen starting temperature and $\alpha \in (0, 1)$ is the cooling constant which indicates the rate of change. For the best performance $\alpha$ should be chosen close to 1, e.g., $\alpha=0.99.$ A typical stopping criterion is $k = k_{max}$ (i.e., whenever the algorithm reaches a predefined maximum number of iterations).
\paragraph{Statement of Algorithm}
As stated in \cite[290]{Henderson2003}, the simulated annealing algorithm consists of the following stages:
\begin{algorithm}
\caption{\textit{Simulated Annealing Algorithm}}\label{Pseudocode_SA}
\begin{algorithmic}
\State \textbf{Select} an initial solution $x \in \Omega$.
\State \textbf{Select} the temperature change counter $k=0$
\State \textbf{Select} a temperature cooling schedule, $t_{k}$ 
\State \textbf{Select} an initial temperature $T = t_{0} \geq 0$ 
\vspace{0.03cm}
\State \textbf{repeat}\\
\begin{enumerate}
    \item Generate a solution $x'\in S$ 
    \item Calculate $\Delta f = f(x')-f(x)$
    \item If $\Delta f \leq 0,$ then \textit{Update.} $x \leftarrow x'$
    \item If $\Delta f > 0,$ then \textit{Update.} $x \leftarrow x'$ with probability $\exp(-\Delta f / t_{k})$
    \item $k \leftarrow k + 1$
\end{enumerate}
\State \textbf{until} stopping criterion is met.
\end{algorithmic}
\end{algorithm}\\
\newpage
\subsubsection{Particle Swarm Optimization}\label{PSO_s}
\paragraph{History and motivation}
Particle Swarm Optimization is an algorithm capable of minimizing a non-convex and multidimensional problem. The algorithm and its concept of "Particle Swarm Optimization (PSO)" were introduced by James Kennedy and Russel Ebhart in 1995 \cite{Kennedy1995}. As described in \cite[1]{PSO}, the algorithm is inspired by reproducing the behaviors of animals in their natural habitat, such as bird flocking or fish schooling. PSO simulates the flocking behavior of birds. The analogy is the following: each bird in a bird flock, searches for the best place to rest (the best place can be characterized by food access, water access, etc.) Birds share this information and as a result, the flock can discover the best place to rest. The algorithm starts with a finite number of initial vectors (particles) which are distributed randomly throughout the domain (search space). The algorithm is based on the following two ideas:
\begin{itemize}
    \item Each particle determines how good its current position is. The lower the function evaluation at the position of the particle is, the better its position is. The particle makes its next move according to the knowledge of its previous positions and the knowledge of the best position obtained by other particles in the swarm. 
    \item In each iteration of the algorithm, a stochastic factor determines, whether a particle moves more towards the globally found best solution or remains more in the vicinity of its own best solution. This stochasticity combined with a good initial distribution of the swarm enables for a thorough exploration of the search space and gives a high chance of finding better solutions close to the global minimum.
\end{itemize}
\paragraph{General setting}
The primary goal of the algorithm is to minimize a function $f: \mathbb{R}^n \longrightarrow \mathbb{R}.$ The algorithm starts by choosing a maximum number of particles $N$ and determines $f(x^{i}) = f(x_{1}^{i},\ldots,x_{n}^{i})$ for each particle $i \in \{1,\ldots,N\}.$ These function evaluations determine how good the positions of each particle $x^{i}$ are.
The swarm, which consists of all particles combined, is interconnected. This means that the particles exchange information, and every particle is aware of the best position that any other particle in the swarm has ever reached. For each particle $i$, its next position is obtained by adding a velocity vector to its current position. As described in \cite[2]{PSO}, the iteration reads as follows:
\begin{gather*}\tag{5.6.3.1}
x^{i}(k + 1) = x^{i}(k) + v^{i}(k+1)\hspace{3.5cm}\\
\rule{0pt}{3.5em} % Adjust the second value to change the spacing
\begin{aligned}
v^{i}(k+1) &= v^{i}(k) \hspace{6.095cm}\textit{1st component}\\ 
&+ C_{1} \cdot Rnd(0, 1) \cdot \left(pb^{i}(k) - x^{i}(k)\right) \hspace{2cm}\textit{2nd component}\\
&+ C_{2} \cdot Rnd(0, 1) \cdot \left(gb^{i}(k) - x^{i}(k)\right) \hspace{2cm}\textit{3rd component}
\end{aligned}
\end{gather*}
\textbf{Caption:}\\
\rule{0pt}{4.5em} % Adjust the second value to change the spacing
\begin{tabular}{ll}
$x^{i}(k)$ & position of particle $i$ at iteration $k$\\
$v^{i}(k)$ & velocity of particle $i$ at iteration $k$\\
$C_{1}$ & cognitive acceleration constant\\
$C_{2}$ & social acceleration constant\\
$Rnd(0, 1)$ & stochastic component of the algorithm, a random number between 0 and 1\\
$pb^{i}(k)$ & personal best position of particle $i$ at iteration $k$\\
$gb^{i}(k)$ & global best position at iteration $k$
\end{tabular}\\
\newline
\newline
Regarding the velocity update, the first component represents the momentum of the particle which is its current velocity. The second component is the cognitive component which depends heavily on the distance of the particle's current position to the best solution it has ever visited. The third component is the social component, which depends heavily on the distance of the particle's current position to the global best position obtained by any particle. At the beginning of the algorithm, the constants $C_{1}$, $C_{2}$, the size of the swarm $N$ and the maximum iteration number $k_{max}$ are initialized. Subsequently, an initial distribution of the swarm over the search space is determined randomly. Additionally, each particle $i$ is randomly assigned an initial velocity vector $v^{i}(0)$. However, to prevent large initial offsets, keeping the initial velocity vector $v^{i}(0)$ relatively low is recommended. The algorithm terminates whenever the maximum number of iterations $k_{max}$ is reached. 
























